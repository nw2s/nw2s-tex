\documentclass{exam}

\RequirePackage{nw2s}

\title{Example Assignment}
\author{Thomas Scott Wilson}
\date{May 2025}

\begin{document}
    \raggedright
    \pointformat{}
    \header{Thomas Scott Wilson}{EEO 101}{Assignment 1}
    \thispagestyle{plain}
    \setlength{\jot}{3ex}

    \begin{titlepage}

        \centering
        \vspace*{\baselineskip}
        Assignment 01\\
        \medskip
        12 May 2025\\
        \vfill
        Thomas Scott Wilson\\
        \medskip
        ID 1234567890 \\
        \vfill
        EEO 101\\
        \medskip
        Example Class\\
        \medskip
        Spring 2025 \\
        \medskip
        Professor Richard Feynman \\
        \vfill

    \end{titlepage}


    \newpage
    \begin{questions}
        \nopointsinmargin

        \question
        This is the first question of the assignment.
        Lorem ipsum dolor sit amet, consectetur adipiscing elit.
        Mauris vel lobortis eros, sed fringilla quam.
        Donec ut condimentum nibh, non scelerisque mi.
        Aliquam consectetur vel ligula sed vehicula.

        \begin{parts}
            \part
            This is part 1.
            Vestibulum pharetra venenatis mi, eget ultricies nisl ultrices nec.
            Sed nulla metus, ornare ut odio eu, dapibus vulputate ante.
            Lorem ipsum dolor sit amet, consectetur adipiscing elit.
            In vel venenatis mi, ornare commodo sem.
            Suspendisse ac sem nibh.

            Note: Formulas should be left aligned with a slight indent.
            They should not be centered.
            \begin{gather*}
                f(x) = a_0 + \sum_{n=1}^{\infty} \left( a_n \cos\left(\frac{2\pi n x}{T}\right) + b_n \sin\left(\frac{2\pi n x}{T} \right) \right) \\
                \text{where} \\
                \quad a_0 = \frac{1}{T} \int_{x_0}^{x_0+T} f(x) \, dx,   \\
                \quad a_n = \frac{2}{T} \int_{x_0}^{x_0+T} f(x) \cos\left( \frac{2\pi n x}{T} \right) dx,   \\
                \quad b_n = \frac{2}{T} \int_{x_0}^{x_0+T} f(x) \sin\left( \frac{2\pi n x}{T} \right) dx  \\
            \end{gather*}
            \part
            Or, to put it another way:
            \begin{gather*}
                f(x) = \sum_{n=-\infty}^{\infty} c_n \, e^{i \frac{2\pi n x}{T}}   \\
                \text{where}   \\
                \quad c_n = \frac{1}{T} \int_{x_0}^{x_0 + T} f(x) \, e^{-i \frac{2\pi n x}{T}} \, dx
            \end{gather*}
        \end{parts}

    \end{questions}


\end{document}
